\documentclass{beamer}

\usepackage{graphicx}
\usepackage[latin1]{inputenc}
\usepackage[T1]{fontenc}
\usepackage[english]{babel}
\usepackage{listings}
\usepackage{xcolor}
\usepackage{eso-pic}
\usepackage{mathrsfs}
\usepackage{url}
\usepackage{amssymb}
\usepackage{amsmath}
\usepackage{multirow}
\usepackage{hyperref}
\usepackage{booktabs}
% \usepackage{bbm}
\usepackage{cooltooltips}
\usepackage{colordef}
\usepackage{beamerdefs}
\usepackage{lvblisting}

\usepackage{multimedia}
\usepackage{algorithmicx}
\usepackage[noend]{algpseudocode}
\usepackage{algorithm}

\pgfdeclareimage[height=2cm]{logobig}{hulogo}
\pgfdeclareimage[height=0.7cm]{logosmall}{Figures/LOB_Logo}

\renewcommand{\titlescale}{1.0}
\renewcommand{\titlescale}{1.0}
\renewcommand{\leftcol}{0.6}

\title[Eigenvalue Problems - Numerical Solutions]{Eigenvalues and Eigenvectors}
\authora{Thomas Siskos}
\authorb{}
\authorc{}

\def\linka{http://github.com/thsis/NIS18}
\def\linkb{}
\def\linkc{}

\institute{Numerical Introductory Seminar \\
Humboldt--Universit�t zu Berlin \\}

\hypersetup{pdfpagemode=FullScreen}

\begin{document}

% 0-1
%%%%%%%%%%%%%%%%%%%%%%%%%%%%%%%%%%%%%%%%
\frame[plain]{

\titlepage
}
\frame{
  \tableofcontents
}
%%%%%%%%%%%%%%%%%%%%%%%%%%%%%%%%%%%%%%%%
\section{Motivation}
%%%%%%%%%%%%%%%%%%%%%%%%%%%%%%%%%%%%%%%%

% 1-1
%%%%%%%%%%%%%%%%%%%%%%%%%%%%%%%%%%%%%%%%
\frame{
\frametitle{Motivation}

}

\subsection{PCA}
\frame{
\frametitle{PCA}

}

\subsection{LDA}
\frame{
\frametitle{LDA}

}

\section{Key Idea \& Definitions}
\frame{
\frametitle{}

}
\subsection{Characteristic Polynomial \& Diagonal Matrices}
\frame{

}
\subsection{Similarity Transformations}
\subsubsection{Householder Reflections}
\subsubsection{Givens Rotations}

\section{Algorithms}
\subsection{Jacobi-Method}
\frame{

}

\frame{
\frametitle{Jacobi-Method}
\begin{algorithm}[H]
\caption{\texttt{jacobi}}
\label{j-meth}
\begin{algorithmic}
  \Require symmetric matrix $A$
  \Ensure $0 < precision < 1$
  \Statex \textbf{initialize: } $L \gets A$; $U \gets I$; $L_{max} \gets 1$
  \While{$L_{max} > precision$}
    \State Find indices $i$, $j$ of largest value in lower triangle of $abs(L)$ 
        \State $L_{max} \gets L_{i,j}$
            \State $\alpha \gets \frac{1}{2}\cdot \arctan(\frac{2A_{i, j}}{A_{i, i}-A_{j, j}})$
    \State $V \gets I$
    \State $V_{i, i}, V_{j, j} \gets \cos \alpha$; $V_{i, j}, V_{j, i} \gets -\sin \alpha, \sin \alpha$
    \State $A \gets V^{\prime} A V$; $U \gets UV$
    
  \EndWhile
  \Return $diag(A), U$
\end{algorithmic}
\end{algorithm}
}
\subsection{QR-Method}
\frame{

}

\subsubsection{Basic Variant}
\frame{
\frametitle{Basic QR-Method}
\begin{algorithm}[H]
\caption{\texttt{QRM1}}
\label{qr1-meth}
  \begin{algorithmic}
    \Require square matrix $A$
    \Statex \textbf{initialize: } $conv \gets False$
    \While{not $conv$}
      \State $Q, R \gets$ QR-Factorization of $A$
      \State $A \gets RQ$
      \If{$A$ is diagonal}
        \State $conv \gets \texttt{True}$
        \Statex
      \EndIf
    \EndWhile  
    \Return $diag\left(A\right), Q$
  \end{algorithmic}
\end{algorithm}
}
\subsubsection{Hessenberg Variant}
\frame{
\frametitle{Refined QR-Method}
\begin{algorithm}[H]
\caption{\texttt{QRM2}}
\label{qr2-meth}
\begin{algorithmic}
  \Require square matrix $A$
  \State $A \gets \texttt{hessenberg(}A\texttt{)}$
  \State continue with: \Call {QRM1} A 
\end{algorithmic}
\end{algorithm}
}

\frame{
  \frametitle{QRM2 Visualized}
  \begin{center}
    \movie[width=8cm, height=5.35cm]{\includegraphics[width=8cm, height=5.35cm]{Figures/placeholder.jpg}}{Figures/qrm_symmetric.mp4}
  \end{center}
}
\subsubsection{Accelerated Variant}
\frame{
\frametitle{Accelerated QR-Method}
\begin{algorithm}[H]
\begin{algorithmic}
\Require square matrix $A$
\end{algorithmic}
\end{algorithm}
}


\section{Analysis}
\subsection{Accuracy}
\frame{

}
\subsection{Efficiency}
\frame{

}


\end{document}
